\documentclass[11pt]{article}

%% Language and font encodings
\usepackage[english]{babel}
\usepackage[T1]{fontenc}
\usepackage{newpxtext}

%% Sets page size and margins
\usepackage[a4paper,top=3cm,bottom=3cm,left=3cm,right=3cm,marginparwidth=2cm]{geometry}

%% Useful packages
\usepackage{amsmath, amsthm, amssymb, mathrsfs}
\usepackage{enumerate, braket}
\usepackage{xcolor, graphicx, tikz}
\usetikzlibrary{cd}
%\usepackage[colorinlistoftodos]{todonotes}
\usepackage[colorlinks=true]{hyperref}
\hypersetup{urlcolor=magenta, linkcolor=blue, citecolor=red}

\numberwithin{equation}{section}

\theoremstyle{plain}
\newtheorem{theorem}{Theorem}[section]
\newtheorem{proposition}[theorem]{Proposition}
\newtheorem{lemma}[theorem]{Lemma}
\newtheorem{corollary}{Corollary}[theorem]

\theoremstyle{definition}
\newtheorem{exercise}{Exercise}[section]
\newtheorem{definition}{Definition}[section]
\newtheorem{example}{Example}[section]
\theoremstyle{remark}
\newtheorem{remark}{Remark}[section]

%%% Math operators
\DeclareMathOperator{\dist}{dist}
\DeclareMathOperator{\rg}{rg}

%%% Shortcuts
\newcommand{\p}{\partial}
\newcommand{\norm}[1]{\left\lVert #1 \right\rVert}

\newcommand{\NN}{\mathbb{N}}
\newcommand{\ZZ}{\mathbb{Z}}
\newcommand{\QQ}{\mathbb{Q}}
\newcommand{\RR}{\mathbb{R}}
\newcommand{\CC}{\mathbb{C}}

\title{\textbf{Interpolation Theory: a Beginner's Guide}}
\author{Jonathan Mui}
\date{last updated: 20 April 2023}

\begin{document}

\maketitle

\tableofcontents

\section{General Properties of Interpolation Spaces}
Main reference: Bergh \& L\"{o}fstr\"{o}m.

\subsection{Categorical notions}
A \emph{category} $\mathscr{C}$ consists of \emph{objects} $A,B,C,\ldots$ and \emph{morphisms}, which are mappings between objects. We write $T:A\to B$ for the morphism $T$ that maps $A$ to $B$. Given morphisms $T:A\to B$ and $S:B\to C$, there is a morphism $ST$, the composition of $S$ with $T$, such that $ST:A\to C$. The composition is associative: if $R,S,T$ are morphisms, then
\begin{equation*}
    T(SR) = (TS)R.
\end{equation*}
Finally, for every object $A\in\mathscr{C}$, there exists an \emph{identity} morphism $I_A$ such that for all morphisms $T:A\to A$, it holds that
\begin{equation*}
    TI_A = I_AT = T.
\end{equation*}
The morphisms in a category are typically `structure preserving maps'. Thus, for example, in the category \textbf{Top} of topological spaces, the morphisms are continuous maps. In these notes, we work almost exclusively with the category of normed vector spaces \textbf{NVS}. In this setting, the morphisms in \textbf{NVS} are \emph{bounded linear operators}.

Let $\mathscr{C}_1, \mathscr{C}$ be two categories. A \emph{(covariant) functor} $F:\mathscr{C}_1 \to \mathscr{C}$ is a rule that assigns to every object $A\in\mathscr{C}_1$ an object $F(A)\in\mathscr{C}$, and to every morphism $T\in\mathscr{C}_1$ there corresponds a morphism $F(T)\in \mathscr{C}$. This correspondence satisfies
\begin{enumerate}[(i)]
    \item $F(I_A) = I_{F(A)}$;
    \item If $T:A\to B$ in $\mathscr{C}_1$, then $F(T):F(A)\to F(B)$ in $\mathscr{C}$; and 
    \item $F(ST)=F(S)F(T)$ for all morphisms $S,T\in\mathscr{C}_1$.
\end{enumerate}

If $\mathscr{C}_1$ is a category such that every object and every morphism in $\mathscr{C}_1$ is respectively an object and morphism in $\mathscr{C}$, then we say that $\mathscr{C}_1$ is a \emph{subcategory} of $\mathscr{C}$ if $F(A)=A$ and $F(T)=T$ defines a functor from $\mathscr{C}_1$ to $\mathscr{C}$.

\subsection{Couples of normed vector spaces}
Let us recall the following fundamental fact about Banach spaces.

\begin{lemma}
\label{lem:abs-converge}
Let $E$ be a normed vector space. Then $E$ is complete (i.e.\ it is a Banach space) if and only if every absolutely convergent series is convergent, i.e.\ $\sum_{n=1}^\infty \|x_n\|<\infty$ implies that there exists $x\in E$ such that $\lim_{n\to\infty}\sum_{k=1}^n x_k = x$ in $E$.
\end{lemma}

\begin{definition}
Let $A_0, A_1$ be topological vector spaces (TVS). We say that $A_0$ and $A_1$ are \emph{compatible} if there is a (Hausdorff) topological vector space $X$ such that $A_0, A_1$ are subspaces of $X$. Then the sum $A_0+A_1$ and intersection $A_0\cap A_1$ are well-defined.
\end{definition}

\begin{remark}
The Hausdorff condition appears in Bergh \& L\"{o}fstr\"{o}m, but according to the conventions of Rudin, for example, all TVS' are Hausdorff already.

In many applications, we will have $A_1 \hookrightarrow A_0$, so we may take $X=A_0$. Then clearly $A_0+A_1=A_0$ and $A_0\cap A_1=A_1$.
\end{remark}

\begin{lemma}
\label{lem:compatible}
Let $A_0, A_1$ be compatible normed vector spaces. Then $A_0\cap A_1$ and $A_0+A_1$ are normed vector spaces with respective norms
\begin{align*}
    \|x\|_{A_0\cap A_1} &= \max\{ \|x\|_{A_0}, \|x\|_{A_1} \} \\
    \|x\|_{A_0+A_1} &= \inf\{ \|x_0\|_{A_0}+\|x_1\|_{A_1} : x=x_0+x_1 \}.
\end{align*}
If $A_0$ and $A_1$ are complete, then $A_0\cap A_1$ and $A_0+A_1$ are also complete.
\end{lemma}
\begin{proof}
It is clear that $A_0\cap A_1$ and $A_0+A_1$ are vector spaces. We verify the triangle inequality. If $x,y\in A_0\cap A_1$, then observe that
\begin{equation*}
    \|x+y\|_{A_i} \le \|x\|_{A_i} + \|y\|_{A_i} \le \|x\|_{A_0\cap A_1} + \|y\|_{A_0\cap A_1}
\end{equation*}
for $i=0,1$. Thus the functional $\|\cdot\|_{A_0\cap A_1}$ defines a norm on $A_0\cap A_1$ (the other properties of the norm are obvious). If $x,y\in A_0+A_1$, then there exist $x_i, y_i \in A_i$ respectively for $i=0,1$ such that $x=x_0+x_1$ and $y=y_0+y_1$. Hence
\begin{equation*}
    \|x+y\|_{A_0+A_1} \le \|x_0+y_0\|_{A_0} + \|x_1+y_1\|_{A_1} \le (\|x_0\|_{A_0}+\|x_1\|_{A_1}) + (\|y_0\|_{A_0}+\|y_1\|_{A_1}).
\end{equation*}
Upon taking the infimum over all possible decompositions $x=x_0+x_1$ and $y=y_0+y_1$, we obtain $\|x+y\|_{A_0+A_1}\le\|x\|_{A_0+A_1}+\|y\|_{A_0+A_1}$, so $\|\cdot\|_{A_0+A_1}$ defines a norm on $A_0+A_1$.

Now assume $A_0,A_1$ are complete. In particular, these are closed subspaces of $X$, a Hausdorff topological vector space. Hence $A_0\cap A_1$ is closed in $X$, and thus closed in the subspace topology of $A_0$ (or $A_1$). It follows that $A_0\cap A_1$ is complete. To prove the completeness of $A_0+A_1$, we use Lemma~\ref{lem:abs-converge}. Assume that $(x_n)$ is a sequence in $A_0+A_1$ such that
\begin{equation*}
    \sum_{n=1}^\infty \|x_n\|_{A_0+A_1} < \infty.
\end{equation*}
Then for each $n$ we may find a decomposition $x_n=x_n^0+x_n^1$, and $\|x_n^0\|_{A_0}+\|x_n^1\|_{A_1} \le 2\|x_n\|_{A_0+A_1}$. It follows that
\begin{equation*}
    \sum_{n=1}^\infty \|x_n^0\|_{A_0} < \infty, \quad \sum_{n=1}^\infty \|x_n^1\|_{A_1} < \infty.
\end{equation*}
By the completeness of $A_0$ and $A_1$, $\sum x_n^0$ converges to some $x^0$ in $A_0$, and $\sum x_n^1$ converges to some $x^1$ in $A_1$. Setting $x=x^0+x^1 \in A_0+A_1$, we find
\begin{equation*}
    \norm{x-\sum_{n=1}^N x_n}_{A_0+A_1} \le \norm{x^0-\sum_{n=1}^N x_n^0}_{A_0} + \norm{x^1-\sum_{n=1}^N x_n^1}_{A_1}
\end{equation*}
and therefore $\sum_{n=1}^N x_n$ converges to $x$ in $A_0+A_1$ as $N\to\infty$. The conclusion follows from Lemma~\ref{lem:abs-converge}.
\end{proof}

Let $\mathscr{C}$ denote any subcategory of \textbf{NVS}. We consider $\mathscr{C}_1$ to be the category of \emph{compatible couples} $(A_0,A_1)$ in $\mathscr{C}$, i.e.\ all pairs of spaces $A_0,A_1\in\mathscr{C}$ such that $A_0\cap A_1$ and $A_0+A_1$ are both objects in $\mathscr{C}$. The morphisms $T=(T_{A_0},T_{A_1})$ in this category are all bounded linear operators from $A_0+A_1$ to $B_0+B_1$ such that
\begin{equation*}
    T_{A_0}:A_0\to B_0 \quad \text{and} \quad T_{A_1}:A_1\to B_1
\end{equation*}
are both morphisms in $\mathscr{C}$, where $T_A$ denotes the restriction of $T$ to $A$. However, in the sequel, we will drop the subscript and just write $T$ to stand for restrictions to various subspaces of $A_0+A_1$. We also write $\|T\|_{A,B}$ for the operator norm of $T:A\to B$.

If $x=x_0+x_1$ with $x_0\in A_0$ and $x_1\in A_1$ as above, then
\begin{equation*}
    \|Tx\|_{B_0+B_1} \le \|T\|_{A_0,B_0}\|x_0\|_{A_0} + \|T\|_{A_1,B_1}\|x_1\|_{A_1}.
\end{equation*}
Consequently
\begin{equation*}
    \|T\|_{A_0+A_1, B_0+B_1} \le \max\{\|T\|_{A_0,B_0}, \|T\|_{A_1,B_1}\},
\end{equation*}
and similarly, it is straightforward to verify
\begin{equation*}
    \|T\|_{A_0\cap A_1, B_0\cap B_1} \le \max\{\|T\|_{A_0,B_0}, \|T\|_{A_1,B_1}\}.
\end{equation*}
We can thus define two basic functors from $\mathscr{C}_1$ to $\mathscr{C}$.

\begin{definition}
We define the functors $\Sigma$ (sum) and $\Delta$ (intersection) by $\Sigma(T)=\Delta(T)=T$ for each morphism $T\in\mathscr{C}_1$, and
\begin{equation*}
    \Sigma(\bar{A}) = A_0+A_1 \quad \text{and} \quad \Delta(\bar{A})=A_0\cap A_1.
\end{equation*}
for every $\bar{A}=(A_0,A_1)\in\mathscr{C}_1$.
\end{definition}

\begin{example}
Let $\mathscr{C}$ be the category of Banach spaces. By Lemma~\ref{lem:compatible}, we may take $\mathscr{C}_1$ to be the category of compatible pairs of Banach spaces, since if $A_0,A_1$ are compatible, then $A_0\cap A_1$ and $A_0+A_1$ are Banach spaces.
\end{example}

\subsection{Definition of interpolation spaces}
In this section, $\mathscr{C}$ will denote any subcategory of \textbf{NVS} such that $\mathscr{C}$ is closed under sum and intersection. As above, we denote by $\mathscr{C}_1$ the category of all compatible pairs of spaces in $\mathscr{C}$.

\begin{definition}
For any pair $\bar{A}=(A_0,A_1)\in\mathscr{C}_1$, the space $A\in\mathscr{C}$ is called an \emph{intermediate space} between $A_0$ and $A_1$ if
\begin{equation*}
    \Delta(\bar{A}) \subseteq A \subseteq \Sigma(\bar{A})
\end{equation*}
with continuous inclusions. The space $A$ is called an \emph{interpolation space} (between $A_0$ and $A_1$) if in addition to being an intermediate space, we have
\begin{equation*}
    T:\bar{A}\to\bar{A} \implies T:A\to A.
\end{equation*}
\end{definition}

We can extend the definition in the following way.
\begin{definition}
Let $\bar{A},\bar{B}$ be two couples in $\mathscr{C}_1$. We say that two spaces $A,B\in\mathscr{C}$ are \emph{interpolation spaces} with respect to $\bar{A}$ and $\bar{B}$ if $A,B$ are intermediate spaces with respect to $\bar{A},\bar{B}$ respectively, and if
\begin{equation*}
    T:\bar{A}\to\bar{B} \implies T:A\to B.
\end{equation*}
More explicitly, this means that if $T:A_0\to B_0$ and $T:A_1\to B_1$, then $T:A\to B$.
\end{definition}

\begin{remark}
\begin{enumerate}[(i)]
    \item If $A,B$ are interpolation spaces with respect to $\bar{A},\bar{B}$, then in general it does \emph{not} follow that $A$ is an interpolation space with respect to $\bar{A}$, nor that $B$ is an interpolation space with respect to $\bar{B}$. [\emph{Comment}: I can't be bothered at the moment to look through Aronszajn \& Gagliardo's 60+ page paper for more details on this.]
    
    \item Observe that $\Delta(\bar{A})$ and $\Delta(\bar{B})$ are interpolation spaces with respect to $\bar{A}$ and $\bar{B}$, and the same is true for $\Sigma(\bar{A})$ and $\Sigma(\bar{B})$.
\end{enumerate}
\end{remark}

If $A=\Delta(\bar{A})$ (or $\Sigma(\bar{A})$) and $B=\Delta(\bar{B})$ (or $\Sigma(\bar{B})$), then it holds that
\begin{equation}
\label{eq:exact-interp}
    \|T\|_{A,B} \le \max\{\|T\|_{A_0,B_0}, \|T\|_{A_1,B_1}\}
\end{equation}
as shown in the previous subsection. In general, if~\eqref{eq:exact-interp} holds, then we say that $A$ and $B$ are \emph{exact interpolation spaces}. However, in many cases, it is only possible to prove
\begin{equation}
\label{eq:unif-interp}
    \|T\|_{A,B} \le C\max\{\|T\|_{A_0,B_0}, \|T\|_{A_1,B_1}\}
\end{equation}
for some constant $C>0$. In this case, we say that $A$ and $B$ are \emph{uniform interpolation spaces}.

The interpolation spaces $A,B$ are said to be \emph{of exponent} $\theta$, $0\le\theta\le 1$, if 
\begin{equation}
\label{eq:theta}
    \|T\|_{A,B} \le C\|T\|^{1-\theta}_{A_0,B_0}\|T\|^\theta_{A_1,B_1}.
\end{equation}
If $C=1$ is possible, then we say that $A,B$ are \emph{exact of exponent} $\theta$. The inequality~\ref{eq:theta} is a convexity result.
\begin{example}
As we will see in the next section, the space $L^p$ is an interpolation space between $L^{p_0}$ and $L^{p_1}$ and exact of exponent $\theta$, provided that
\begin{equation*}
    \frac{1}{p} = \frac{1-\theta}{p_0} + \frac{\theta}{p_1}.
\end{equation*}
This is the assertion of the Riesz-Thorin theorem (Theorem~\ref{thm:Riesz-Thorin}).
\end{example}

\begin{theorem}
\label{thm:Banach-interp}
In the category of Banach spaces, suppose that $A,B$ are interpolation spaces with respect to the couples $\bar{A},\bar{B}$ respectively. Then $A,B$ are uniform interpolation spaces.
\end{theorem}
\begin{proof}
Consider the set $\mathcal{T}$ of all morphisms $T$ in $\mathscr{C}_1$ such that $T:\bar{A}\to\bar{B}$. In particular, this means that $T:A\to B$ is a bounded linear operator. Let $L_1$ denote $\mathcal{T}$ equipped with the norm
\begin{equation*}
    \|T\|_{L_1}:=\max\{\|T\|_{A,B},\|T\|_{A_0,B_0},\|T\|_{A_1,B_1}\},
\end{equation*}
and let $L_2$ denote $\mathcal{T}$ equipped with the norm
\begin{equation*}
    \|T\|_{L_2}:=\max\{\|T\|_{A_0,B_0},\|T\|_{A_1,B_1}\}.
\end{equation*}
It is a straightforward exercise (using the properties of intermediate spaces) to check that $L_1, L_2$ are Banach spaces with their respective norms. Thus the identity map $i:L_1\to L_2$ is a linear, bounded, bijective map. By the open mapping theorem, $i^{-1}:L_2\to L_1$ is also a bounded linear operator, and thus there exists a constant $C>0$ such that
\begin{equation*}
    \|T\|_{A,B} \le \|T\|_{L_1} \le C\|T\|_{L_2}
\end{equation*}
for all $T\in\mathcal{T}$.
\end{proof}

\subsection{Interpolation functors}

One of the fundamental tasks in interpolation theory is the actual construction of interpolation spaces.
\begin{definition}
An \emph{interpolation functor} (or \emph{interpolation method}) on $\mathscr{C}$ is a functor $F:\mathscr{C}_1\to\mathscr{C}$ such that if $\bar{A},\bar{B}$ are couples in $\mathscr{C}_1$, then $F(\bar{A})$ and $F(\bar{B})$ are interpolation spaces with respect to $\bar{A}$ and $\bar{B}$. Moreover it holds that
\begin{equation*}
    F(T)=T \qquad \text{for all } T:\bar{A}\to \bar{B} \text{ in }\mathscr{C}_1.
\end{equation*}
We say that $F$ is a \emph{uniform} (resp.\ \emph{exact}) interpolation functor if $F(\bar{A})$ and $F(\bar{B})$ are uniform (resp.\ \emph{exact}) interpolation spaces with respect to $\bar{A}$ and $\bar{B}$. Similarly we say that $F$ is \emph{(exact) of exponent} $\theta$ if $F(\bar{A}),F(\bar{B})$ are (exact) of exponent $\theta$.
\end{definition}

Let $\bar{A}=(A_0,A_1)$ and $\bar{B}=(B_1,B_0)$ be couples in $\mathscr{C}$, and suppose $F$ is an interpolation functor. The situation is best remembered by the following diagram:
\begin{center}
\begin{tikzcd}
B_0 \arrow[r, hook]                & F(\bar{B}) \arrow[r, hook]                & B_1                \\
A_0 \arrow[u, "T"] \arrow[r, hook] & F(\bar{A}) \arrow[u, "T"] \arrow[r, hook] & A_1 \arrow[u, "T"]
\end{tikzcd}
\end{center}

\begin{example}
The simplest interpolation functors are of course $\Delta$ and $\Sigma$. These functors are exact (recall~\eqref{eq:exact-interp}) on any admissible subcategory $\mathscr{C}$ of the category of normed vector spaces.
\end{example}

\begin{remark}
By Theorem~\ref{thm:Banach-interp}, any interpolation functor in the category of Banach spaces is uniform. In particular, we have the estimate
\begin{equation*}
    \|T\|_{F(\bar{A}),F(\bar{B})} \le C\max\{\|T\|_{A_0,B_0}, \|T\|_{A_1,B_1}\}
\end{equation*}
for some constant $C>0$ possibly dependent on the couples $\bar{A},\bar{B}$. In general, if $C$ can be chosen independently of the couples, we say that $F$ is a \emph{bounded interpolation functor}.
\end{remark}

\section{The Riesz-Thorin theorem}
Main reference: Grafakos, \emph{Classical Fourier Analysis}.

\begin{lemma}[Hadamard `three-lines lemma']
\label{lem:three-lines}
Let $F$ be analytic in the open strip $S=\{z\in\CC:0<\mathrm{Re}(z)<1\}$, and continuous and bounded on the closure $\bar{S}$. Assume moreover that $|F(z)|\le B_0$ when $\mathrm{Re}(z)=0$ and $|F(z)|\le B_1$ when $\mathrm{Re} (z)=1$ for some constants $0<B_0,B_1<\infty$. Then
\begin{equation*}
    |F(z)| \le B_0^{1-\theta}B_1^\theta
\end{equation*}
when $\mathrm{Re}(z)=\theta$, for any $\theta\in [0,1]$.
\end{lemma}
\begin{proof}
Since $B_0, B_1>0$, the functions $B_0^{1-z}=B_0\exp(-z\log B_0)$ and $B_1^z=\exp(z\log B_1)$ are analytic in $S$. For all $z=x+iy\in \bar{S}$, we also observe the uniform lower bounds
\begin{equation*}
    |B_0^{1-z}| = B_0 e^{-x\log B_0} \ge \min\{1,B_0\} \qquad |B_1^z| = e^{x\log B_1} \ge \min\{1,B_1\}.
\end{equation*}
Hence
\begin{equation*}
    G(z):=F(z)(B_0^{1-z}B_1^z)^{-1}
\end{equation*}
is analytic in $S$. Let $G_n(z):=G(z)e^{(z^2-1)/n}$ for each $n\in\NN$. Since $F$ is bounded on $\bar{S}$ and $|B_0^{1-z}B_1^z| > 0$ for all $z\in\bar{S}$ as shown above, we conclude that $|G|$ is bounded by some constant $M$ in $\bar{S}$. Since
\begin{equation*}
    |G_n(x+iy)| \le Me^{-y^2/n}e^{(x^2-1)/n} \le Me^{-y^2/n},
\end{equation*}
we find that for fixed $n\in\NN$, $G_n(x+iy) \to 0$ uniformly for $x\in [0,1]$ as $|y|\to \infty$. Now choose $R=R(n)>0$ large enough so that $|G_n|\le 1$ for all $x\in [0,1]$ and $|y|\ge R$. The assumptions on $F$ also imply that $|G_n|\le 1$ on the two lines $\mathrm{Re}(z)=0$ and $\mathrm{Re}(z)=1$ which form the boundary of $S$. Now the maximum modulus principle implies that $|G_n|\le 1$ in the rectangle $[0,1]\times [-R,R]$, and thus $|G_n|\le 1$ on $\bar{S}$. Then we can take $n\to\infty$ to conclude that $|G|\le 1$ on $\bar{S}$. Finally, setting $z=\theta+it$, we deduce
\begin{equation*}
    |F(\theta+it)| \le |B_0^{1-\theta-it}B_1^{\theta+it}|=B_0^{1-\theta}B_1^\theta
\end{equation*}
which completes the proof.
\end{proof}

\begin{theorem}[Riesz-Thorin interpolation]
\label{thm:Riesz-Thorin}
Let $(X,\mu), (Y,\nu)$ be $\sigma$-finite measure spaces. Suppose $T$ is a linear operator defined for measurable functions on $X$ with values in the measurable functions on $Y$. Let $1\le p_0,p_1,q_0,q_1 \le\infty$ and assume
$T:L^{p_0}(X)\to L^{q_0}(Y)$ is bounded with norm $\|T\|_{p_0\to q_0}=M_0$, and $T:L^{p_1}(X)\to L^{q_1}(Y)$ is bounded with norm $\|T\|_{p_1\to q_1}=M_1$. Then for all $\theta\in [0,1]$ it holds that
\begin{equation}
\label{eq:Tpq}
    \|T\|_{p\to q} \le M_0^{1-\theta}M_1^\theta
\end{equation}
where
\begin{equation}
\label{eq:pq-interp}
    \frac{1}{p}=\frac{1-\theta}{p_0}+\frac{\theta}{p_1}, \qquad \frac{1}{q}=\frac{1-\theta}{q_0}+\frac{\theta}{q_1}.
\end{equation}
\end{theorem}

\begin{remark}
The inequality~\eqref{eq:Tpq} shows that the operator norm $M$, when seen as a `function' of $1/p$, is log-convex (i.e.\ $\log M$ is convex). The equations~\eqref{eq:pq-interp} and thus the result of the Theorem can be easily remembered in the following way: let's say that a point $(1/p,1/q)$ in the plane is a `mapping pair' for $T$ if $T:L^p \to L^q$ is a bounded linear operator. Thus if $X_0=(1/p_0,1/q_0)$ and $X_1=(1/p_1,1/q_1)$ are mapping pairs, then every point in the line segment joining $X_0$ to $X_1$ is also a mapping pair.
\end{remark}

\begin{proof}[Proof of Theorem~\ref{thm:Riesz-Thorin}]
We first recall that
\begin{equation*}
    \|Tf\|_{L^q(Y,\nu)} = \sup\left\{ \left|\int_Y (Tf)(y)g(y)\,\nu(dy)\right| : g\in L^{q'}(Y,\nu), \|g\|_{L^{q'}}\le 1 \right\}
\end{equation*}
(a consequence of the Riesz representation theorem and a standard duality result from functional analysis). It suffices to prove
\begin{equation*}
    \|Tf\|_q \le M_0^{1-\theta}M_1^\theta\|f\|_p
\end{equation*}
for all \emph{simple functions} $f$, that is, $f$ of the form
\begin{equation*}
    f = \sum_{k=1}^m a_k e^{i\alpha_k} \mathbf{1}_{A_k}
\end{equation*}
where $a_k>0, \alpha_k\in\RR$, and the sets $A_k \subset X$ are pairwise disjoint with finite $\mu$-measure. The general case follows immediately, since simple functions are dense in $L^p(X,\mu)$ and $L^q(Y,\nu)$.

Let $f\in L^p(X)$ and $g\in L^{q'}(Y)$ be simple functions, hence
\begin{equation*}
    f = \sum_{k=1}^m a_k e^{i\alpha_k} \mathbf{1}_{A_k}, \quad g = \sum_{j=1}^n b_j e^{i\beta_j} \mathbf{1}_{B_j}
\end{equation*}
where $b_j>0, \beta_j\in\RR$, and the sets $B_j \subset Y$ are pairwise disjoint with finite $\nu$-measure. Now let
\begin{equation*}
    P(z):=\frac{p}{p_0}(1-z) +\frac{p}{p_1}z, \qquad Q(z):=\frac{q'}{q_0'}(1-z) +\frac{q'}{q_1'}z.
\end{equation*}
For $z$ in the closed strip $\bar{S}=\{z\in\CC: 0\le\mathrm{Re}(z)\le 1\}$, we define
\begin{equation*}
    f_z:=\sum_{k=1}^m a_k^{P(z)}e^{i\alpha_k}\mathbf{1}_{A_k}, \qquad g_z:=\sum_{j=1}^n b_j^{Q(z)}e^{i\beta_j}\mathbf{1}_{B_j},
\end{equation*}
and
\begin{equation*}
    F(z):=\int_Y (Tf_z)(y)g_z(y)\,\nu(dy).
\end{equation*}
We have $f_\theta=f$ and $g_\theta=g$ due to~\eqref{eq:pq-interp}. By linearity, we obtain
\begin{equation*}
    F(z)=\sum_{k=1}^m \sum_{j=1}^n a_k^{P(z)}b_j^{Q(z)}e^{i\alpha_k}e^{i\beta_j} \int_Y (T\mathbf{1}_{A_k})(y)\mathbf{1}_{B_j}(y)\,\nu(dy).
\end{equation*}
A simple application of H\"{o}lder's inequality and the assumption that $\|T\|_{p_m\to q_m}=M_m, m\in\{0,1\}$, shows that
\begin{equation*}
    \int_Y (T\mathbf{1}_{A_k}) \mathbf{1}_{B_j} \,d\nu \le M_m \mu(A_k)^{1/p_m}\nu(B_j)^{1/q_m'}, \qquad m\in\{0,1\}.
\end{equation*}
Using the disjointness of the sets $A_k$ and $|a_k^{P(it)}|=a_k^{p/p_0}$, we find
\begin{equation*}
    \|f_{it}\|_{p_0} = \|f\|_{p}^{p/p_0}.
\end{equation*}
(and note that this holds even for $p_0=\infty$). Similarly, the disjointness of the sets $B_j$ and $|b_j^{Q(it)}|=b_j^{q'/q_0'}$ imply
\begin{equation*}
    \|g_{it}\|_{q_0'} = \|g\|_{q'}^{q'/q_0'}
\end{equation*}
(which is valid even for $q_0=1$, i.e.\ $q_0'=\infty$). Thus by H\"{o}lder's inequality and the assumptions, we obtain
\begin{equation}
\label{eq:F-it-bound}
    |F(it)|\le \|Tf_{it}\|_{q_0}\|g_{it}\|_{q_0'}\le M_0\|f_{it}\|_{p_0}\|g_{it}\|_{q_0'}\le M_0\|f\|_p^{p/p_0}\|g\|_{q'}^{q'/q_0'}.
\end{equation}
By similar calculations, we obtain
\begin{equation*}
    \|f_{1+it}\|_{p_1}=\|f\|_p^{p/p_1}, \qquad \|g_{1+it}\|_{q_1'}=\|g\|_{q'}^{q'/q_1'},
\end{equation*}
(valid even for $p_1=\infty$ and $q_1'=\infty$), and therefore
\begin{equation}
\label{eq:F-1+it-bound}
    |F(1+it)|\le M_1\|f\|_p^{p/p_1}\|g\|_{q'}^{q'/q_1'}
\end{equation}
in analogy with~\eqref{eq:F-it-bound}.

Finally, since $a_k,b_j>0$, it follows that $F$ is analytic in $S$, and bounded and continuous on $\bar{S}$. Combining estimates~\eqref{eq:F-it-bound}, \eqref{eq:F-1+it-bound} and Lemma~\ref{lem:three-lines}, we deduce
\begin{equation*}
    |F(z)|\le \left(M_0\|f\|_p^{p/p_0}\|g\|_{q'}^{q'/q_0'}\right)^{1-\theta} \left(M_1\|f\|_p^{p/p_1}\|g\|_{q'}^{q'/q_1'}\right)^\theta = M_0^{1-\theta}M_1^\theta \|f\|_p\|g\|_{q'}
\end{equation*}
when $\mathrm{Re}(z)=\theta$. Since $P(\theta)=Q(\theta)=1$, it follows that
\begin{equation*}
    |F(\theta)|= \left|\int_Y (Tf)g\,d\nu\right| \le M_0^{1-\theta}M_1^\theta \|f\|_p\|g\|_{q'}.
\end{equation*}
Taking the supremum over all simple functions $g\in L^{q'}$ with $\|g\|_{q'}\le 1$, we conclude the proof.
\end{proof}

We now give some applications of the Riesz-Thorin theorem. Let $X=Y=\RR^n$ with $\mu=\nu=dx$, the Lebesgue measure. We consider the Fourier transform $\mathcal{F}$ defined by
\begin{equation*}
    \mathcal{F}[f](\xi):=\frac{1}{(2\pi)^{n/2}}\int f(x)e^{-ix\cdot\xi}\,dx.
\end{equation*}
It is obvious that $\mathcal{F}$ maps $L^1$ into $L^\infty$. Moreover, it is well-known that $\mathcal{F}$ extends to an operator on $L^2$, and Plancherel's theorem states that $\mathcal{F}$ is an isometry on $L^2$. Hence
\begin{align*}
    \mathcal{F}:L^1 \to L^\infty \quad & \text{with norm }\frac{1}{(2\pi)^{n/2}}, \\
    \mathcal{F}:L^2 \to L^2 \quad & \text{with norm }1.
\end{align*}

\begin{theorem}[Hausdorff-Young inequality]
If $1\le p\le 2$, then $\mathcal{F}:L^p\to L^{p'}$ (where $1/p+1/p'=1$ as usual), with the estimate
\begin{equation}
    \|\mathcal{F}f\|_{p'} \le (2\pi)^{-n/p'}\|f\|_p.
\end{equation}
\end{theorem}

\begin{proof}
We apply the Riesz-Thorin theorem with $p_0=q_0=2$ and $p_1=1,q_1=\infty$. Thus $\mathcal{F}:L_p \to L_q$ with
\begin{equation*}
    \frac{1}{p}=\frac{1-\theta}{2}+\frac{\theta}{1}, \quad \frac{1}{q}=\frac{1-\theta}{2}+\frac{\theta}{\infty} \quad (0<\theta<1).
\end{equation*}
It follows that $\frac{1}{p}+\frac{1}{q}=1$, which yields $q=p'$. The norm of $\mathcal{F}:L^p\to L^{p'}$ is therefore bounded above by
\begin{equation*}
    \frac{1}{(2\pi)^{\frac{n(1-\theta)}{2}}}\cdot 1^{\theta} = (2\pi)^{-n/p'}
\end{equation*}
which proves the theorem.
\end{proof}

\begin{theorem}[Young's inequality for convolutions]
If $k\in L^q$ and $f\in L^p$, where $1<p<q'$, then $k*f \in L^r$ for $\frac{1}{r}=\frac{1}{p}-\frac{1}{q'}$ (equivalently $\frac{1}{r}+1=\frac{1}{p}+\frac{1}{q}$), and
\begin{equation*}
    \|k*f\|_r \le \|k\|_q \|f\|_p.
\end{equation*}
\end{theorem}

\begin{proof}
By Minkowski's convolution inequality (see e.g.\ Grafakos Theorem 1.2.10), we have
\begin{equation*}
    \|k*f\|_q \le \|k\|_q \|f\|_1.
\end{equation*}
On the other hand, H\"{o}lder's inequality yields
\begin{equation*}
    \|k*f\|_\infty \le \|k\|_q \|f\|_{q'}.
\end{equation*}
Therefore, by Riesz-Thorin interpolation, we see that the convolution $T:f \mapsto k*f$ is a map $L^p\to L^r$ where
\begin{equation*}
    \frac{1}{p}=\frac{1-\theta}{1}+\frac{\theta}{q'}=1-\frac{\theta}{q}, \quad \frac{1}{r}=\frac{1-\theta}{q}+\frac{\theta}{\infty}.
\end{equation*}
Therefore $\frac{\theta}{q}=1-\frac{1}{p}=\frac{1}{q}-\frac{1}{r}$, which gives $\frac{1}{r}=\frac{1}{p}-(1-\frac{1}{q})=\frac{1}{p}-\frac{1}{q'}$. We conclude
\begin{equation*}
    \|k*f\|_r \le \|k\|_q\|f\|_p
\end{equation*}
as claimed.
\end{proof}

\end{document}
